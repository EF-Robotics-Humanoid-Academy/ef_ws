\documentclass[10pt]{beamer}
\usepackage[utf8]{inputenc}
\usepackage[T1]{fontenc}
\usepackage{graphicx}
\usepackage{xcolor}
\usepackage{listings}
\usepackage{fancyvrb}
\usepackage{hyperref}
\usepackage{booktabs}
\usepackage{array}
\usepackage{verbatim}

% Define custom colors from the CSS
\definecolor{AccentColor2}{HTML}{FA1C63}
\definecolor{AccentColor1}{HTML}{222222}

% Theme settings - SWAPPED ASSIGNMENTS
\usetheme{default}
\setbeamercolor{palette primary}{bg=AccentColor1,fg=white}
\setbeamercolor{palette secondary}{bg=AccentColor2,fg=white}
\setbeamercolor{palette tertiary}{bg=AccentColor2,fg=white}
\setbeamercolor{title}{bg=AccentColor1,fg=white}
\setbeamercolor{frametitle}{bg=AccentColor1,fg=white}
\setbeamercolor{structure}{fg=AccentColor2}
\setbeamertemplate{footline}[frame number]{}
\setbeamercolor{block body}{bg=gray!10}
\setbeamercolor{block title}{use=structure,fg=white,bg=structure.fg!75!black}
\setbeamersize{text margin left=0.6cm,text margin right=0.6cm}
\setbeamertemplate{itemize/enumerate body begin}{\small}
\setbeamertemplate{itemize/enumerate subbody begin}{\small}

\lstset{
    basicstyle=\ttfamily\scriptsize,
    backgroundcolor=\color{gray!10},
    frame=single,
    breaklines=true,
    keywordstyle=\color{blue},
    commentstyle=\color{gray},
    stringstyle=\color{red},
}

% Paths for local images
\graphicspath{{./}{imgs/}{../inhalt/}{../inhalt/assets/}}

\title[Bring robot to a safe stop and balanced stand]{Bring robot to a safe stop and balanced stand}
\author{Unitree G1 Academy}
\date{February 2026}
\logo{\includegraphics[height=1cm]{Logo_1.jpg}}

\begin{document}
\begin{frame}
    \titlepage
\end{frame}

\begin{frame}[plain]
\centering
\vfill
\Huge{Day 2 outline}
\vfill
\end{frame}

\begin{frame}{Day 2 outline}
\begin{itemize}
\item High Level Gait
\item Using IMU Data
\item Lunch break
\item Multi Step motion sequence
\item Troubleshooting
\end{itemize}
Beginner focus:
\begin{itemize}
\item You will build confidence with short, low-speed walking commands.
\end{itemize}
\end{frame}

\begin{frame}[plain]
\centering
\vfill
\Huge{Lunch Break}
\vfill
\end{frame}

\begin{frame}[plain]
\centering
\vfill
\Huge{Schedule (Tag 2 -- Di 17.02.2026)}
\vfill
\end{frame}

\begin{frame}[allowframebreaks,t]{Schedule (Tag 2 -- Di 17.02.2026)}
\small
\begin{itemize}
\item 08.00 Crew
\item 08:30 Einfinden der Teilnehmer
\item 09:00 Schulungsblock 1
\item 10:30 Kaffee- und Teepause
\item 10:45 Schulungsblock 2
\item 12:00 Mittagessen
\item 13:00 Schulungsblock 3
\item 14:30 Kaffee- und Teepause mit Snacks
\item 15:00 Schulungsblock 4
\item 17:00 Ende Tag 2
\item 17:00 -- 17:30 Podcast / Testimonials
\item 17:30 -- 17:45 Puffer
\item 17:45 -- 18:15 Tages-Recap
\item 18:15 -- 18:30 Tagesabschluss
\end{itemize}
\end{frame}

\begin{frame}[plain]
\centering
\vfill
\Huge{High Level Gait}
\vfill
\end{frame}

\begin{frame}{High Level Gait}
\begin{center}
\includegraphics[width=0.85\linewidth,height=0.45\textheight,keepaspectratio]{imgs/66d93f622b6a4ce2a962be2dc2c91054_830x986.jpg}
\end{center}
Measurement: define target speed, distance, time, or number of steps before commanding motion.
Beginner notes:
\begin{itemize}
\item Start with low speed and short duration (e.g., 0.1--0.2 m/s for 2--3 seconds).
\item Always send a stop command after each motion segment.
\item No obstacle avoidance is active in low-level velocity control.
\end{itemize}
\end{frame}

\begin{frame}[fragile]{Implementation details (from repo scripts)}
\begin{itemize}
\item \texttt{g1/scripts/basic/g1\_hl\_gait\_measure.py} subscribes to \texttt{rt/sportmodestate}, samples position and \texttt{foot\_force}, and computes total path length plus step count. Contact steps are detected by thresholding each foot force signal and counting rising edges.
\item The script supports \texttt{--no-command} for measurement-only runs, \texttt{--force-threshold} to tune contact detection, and \texttt{--csv} to log time, pose, gait type, and foot force for analysis.
\end{itemize}
\textbf{Parametrized walk example}
\begin{lstlisting}
from unitree_sdk2py.rpc import LocoClient
from unitree_sdk2py.dds import ChannelFactory

ChannelFactory.Instance().Init(0, "enp3s0")

loco = LocoClient()
loco.Init()

loco.SetVelocity(0.2, 0.0, 0.0, 3.0)
loco.StopMove()
\end{lstlisting}
\textbf{Video: Basic motor control (Day 2)}
\begin{center}
\includegraphics[width=0.8\linewidth]{./placeholder.jpg}
\end{center}
Video file: \href{https://doc-cdn.unitree.com/static/2024/9/23/982715b4258a4acda666792ac8c964f6.mp4}{https://doc-cdn.unitree.com/static/2024/9/23/982715b4258a4acda666792ac8c964f6.mp4}
\textbf{Parametrized turn example}
\begin{lstlisting}
from unitree_sdk2py.rpc import LocoClient
from unitree_sdk2py.dds import ChannelFactory

ChannelFactory.Instance().Init(0, "enp3s0")

loco = LocoClient()
loco.Init()

loco.SetVelocity(0.0, 0.0, 0.6, 2.0)
loco.StopMove()
\end{lstlisting}
\end{frame}

\begin{frame}[plain]
\centering
\vfill
\Huge{Using IMU Data}
\vfill
\end{frame}

\begin{frame}{Using IMU Data}
\begin{center}
\includegraphics[width=0.8\linewidth]{./placeholder.jpg}
\end{center}
\begin{itemize}
\item Use IMU data to correct drift and improve motion accuracy in parametrized scripts.
\item Visualize stability state during gait tests to validate balance.
\end{itemize}
Beginner notes:
\begin{itemize}
\item The IMU tells you if the robot is tilting or drifting.
\item If you see large pitch/roll changes, reduce speed and shorten the step duration.
\end{itemize}
\end{frame}

\begin{frame}{Implementation details (from repo scripts)}
\begin{itemize}
\item \texttt{g1/scripts/sensors/g1\_stability\_view.py} subscribes to \texttt{rt/lowstate} for IMU \texttt{rpy} and to \texttt{rt/odom} for position, then plots roll/pitch/yaw, tilt magnitude, and a stability score in real time.
\item The stability score is computed as \texttt{1.0 - (tilt\_deg / max\_tilt\_deg)} and can be tuned with \texttt{--max-tilt-deg}. Use \texttt{--history} to control the rolling window length.
\end{itemize}
\end{frame}

\begin{frame}[plain]
\centering
\vfill
\Huge{Multi Step motion sequence}
\vfill
\end{frame}

\begin{frame}[fragile]{Multi Step motion sequence}
\begin{center}
\includegraphics[width=0.8\linewidth]{./placeholder.jpg}
\end{center}
\begin{lstlisting}
from unitree_sdk2py.rpc import LocoClient, ArmActionClient
from unitree_sdk2py.dds import ChannelFactory

ChannelFactory.Instance().Init(0, "enp3s0")

loco = LocoClient()
loco.Init()

arm = ArmActionClient()
arm.Init()

loco.SetVelocity(0.2, 0.0, 0.0, 3.0)

loco.SetVelocity(0.0, 0.0, 0.6, 1.5)

loco.SetVelocity(0.2, 0.0, 0.0, 3.0)

arm.DoAction(26)

loco.SetVelocity(0.0, 0.0, 0.6, 1.5)

loco.SetVelocity(0.2, 0.0, 0.0, 3.0)

loco.StopMove()
\end{lstlisting}
Beginner notes:
\begin{itemize}
\item Sequence each action clearly: walk, stop, turn, stop, walk.
\item Insert short pauses if the robot looks unstable.
\item Keep arm actions short and simple while walking.
\end{itemize}
\end{frame}

\begin{frame}{Implementation details (from repo scripts)}
\begin{itemize}
\item \texttt{g1/scripts/basic/g1\_hl\_motion\_sequence.py} computes walk and turn durations from \texttt{--walk-m}, \texttt{--walk-v}, \texttt{--turn-deg}, and \texttt{--turn-vyaw}, then sends velocities at a fixed \texttt{--cmd-hz} rate using a \texttt{Move/SetVelocity} loop with explicit \texttt{StopMove} between segments.
\item The script attempts a right-hand wave via whichever arm/hand client exists and skips the wave if no supported method is present.
\end{itemize}
\end{frame}

\begin{frame}[plain]
\centering
\vfill
\Huge{Troubleshooting}
\vfill
\end{frame}

\begin{frame}[fragile]{Troubleshooting}
\small
\begin{center}
\includegraphics[width=0.85\linewidth,height=0.45\textheight,keepaspectratio]{imgs/236fa93a8fae4eaa8815004f42e87ede_1065x1419.jpg}
\end{center}
\textbf{Recovery procedure (balanced stand)}
\begin{lstlisting}
from unitree_sdk2py.rpc import LocoClient
from unitree_sdk2py.dds import ChannelFactory

ChannelFactory.Instance().Init(0, "enp3s0")

loco = LocoClient()
loco.Init()

loco.StopMove()
\end{lstlisting}
\textbf{Intentional error handling}
\begin{itemize}
\item Try an invalid parameter range and observe errors.
\item Correct parameter ranges and retry.
\item Confirm debug mode before low-level commands.
\end{itemize}
Beginner notes:
\begin{itemize}
\item If the robot does not move, verify network interface and debug mode.
\item If motion is jerky, reduce speed and duration first, then retry.
\end{itemize}
\end{frame}

\end{document}