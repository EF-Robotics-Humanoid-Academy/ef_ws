\documentclass[10pt]{beamer}
\usepackage[utf8]{inputenc}
\usepackage[T1]{fontenc}
\usepackage{graphicx}
\usepackage{xcolor}
\usepackage{listings}
\usepackage{fancyvrb}
\usepackage{hyperref}
\usepackage{booktabs}
\usepackage{array}
\usepackage{verbatim}

% Define custom colors from the CSS
\definecolor{AccentColor2}{HTML}{FA1C63}
\definecolor{AccentColor1}{HTML}{222222}

% Theme settings - SWAPPED ASSIGNMENTS
\usetheme{default}
\setbeamercolor{palette primary}{bg=AccentColor1,fg=white}
\setbeamercolor{palette secondary}{bg=AccentColor2,fg=white}
\setbeamercolor{palette tertiary}{bg=AccentColor2,fg=white}
\setbeamercolor{title}{bg=AccentColor1,fg=white}
\setbeamercolor{frametitle}{bg=AccentColor1,fg=white}
\setbeamercolor{structure}{fg=AccentColor2}
\setbeamertemplate{footline}[frame number]{}
\setbeamercolor{block body}{bg=gray!10}
\setbeamercolor{block title}{use=structure,fg=white,bg=structure.fg!75!black}
\setbeamersize{text margin left=0.6cm,text margin right=0.6cm}
\setbeamertemplate{itemize/enumerate body begin}{\small}
\setbeamertemplate{itemize/enumerate subbody begin}{\small}

\lstset{
    basicstyle=\ttfamily\scriptsize,
    backgroundcolor=\color{gray!10},
    frame=single,
    breaklines=true,
    keywordstyle=\color{blue},
    commentstyle=\color{gray},
    stringstyle=\color{red},
}

% Paths for local images
\graphicspath{{./}{imgs/}{../inhalt/}{../inhalt/assets/}}

\title[Day 3: Navigation and Environment Awareness]{Day 3: Navigation and Environment Awareness}
\author{Unitree G1 Academy}
\date{February 2026}
\logo{\includegraphics[height=1cm]{Logo_1.jpg}}

\begin{document}
\begin{frame}
    \titlepage
\end{frame}

\begin{frame}[plain]
\centering
\vfill
\Huge{Day 3 outline}
\vfill
\end{frame}

\begin{frame}{Day 3 outline}
\begin{itemize}
\item Complex motion
\item SLAM
\item Lunch break
\item Path planning
\item Obstacle avoidance
\end{itemize}
Beginner focus:
\begin{itemize}
\item We keep navigation goals small and predictable before attempting complex routes.
\end{itemize}
\end{frame}

\begin{frame}[plain]
\centering
\vfill
\Huge{Lunch Break}
\vfill
\end{frame}

\begin{frame}[plain]
\centering
\vfill
\Huge{Schedule (Tag 3 -- Mi 18.02.2026)}
\vfill
\end{frame}

\begin{frame}[allowframebreaks,t]{Schedule (Tag 3 -- Mi 18.02.2026)}
\small
\begin{itemize}
\item 08.00 Uhr Crew
\item 08:30 Einfinden der Teilnehmer
\item 09:00 Schulungsblock 1
\item 10:30 Kaffee- und Teepause
\item 10:45 Schulungsblock 2
\item 12:00 Mittagessen
\item 13:00 Schulungsblock 3
\item 14:30 Kaffee- und Teepause mit Snacks
\item 15:00 Schulungsblock 4
\item 17:00 Ende Tag 3
\item 17:00 -- 17:30 Podcast / Testimonials
\item 17:30 -- 17:45 Puffer
\item 17:45 -- 18:15 Tages-Recap
\item 18:15 -- 18:30 Tagesabschluss
\end{itemize}
\end{frame}

\begin{frame}[plain]
\centering
\vfill
\Huge{Complex motion}
\vfill
\end{frame}

\begin{frame}[fragile]{Complex motion}
\href{https://www.unitree.com/images/7e51cf20dc6145cf99ae0d0b6ea4d2c5.mp4}{https://www.unitree.com/images/7e51cf20dc6145cf99ae0d0b6ea4d2c5.mp4}
\textbf{Hardcoded pick and place (known pose)}
\begin{lstlisting}
from unitree_sdk2py.arm import ArmSdk  # Replace with actual SDK module

arm = ArmSdk(side="right")

pick_pose = {"x": 0.45, "y": -0.10, "z": 0.35, "roll": 0.0, "pitch": 1.57, "yaw": 0.0}
place_pose = {"x": 0.30, "y": 0.20, "z": 0.40, "roll": 0.0, "pitch": 1.57, "yaw": 0.0}

arm.MoveToPose(pick_pose)
arm.CloseGripper()
arm.MoveToPose(place_pose)
arm.OpenGripper()
\end{lstlisting}
Beginner notes:
\begin{itemize}
\item Start with a known object pose and slow arm motion.
\item Keep the robot in a stable stand during the pick-and-place.
\item Use small, testable changes to poses (5--10 cm at a time).
\end{itemize}
\end{frame}

\begin{frame}[plain]
\centering
\vfill
\Huge{SLAM}
\vfill
\end{frame}

\begin{frame}{SLAM}
\begin{center}
\includegraphics[width=0.8\linewidth]{./placeholder.jpg}
\end{center}
\end{frame}

\begin{frame}{SLAM Navigation Service Interface (Beginner Flow)}
\small
Before you start:
\begin{itemize}
\item Ensure your PC and the robot are on the same LAN segment.
\item Turn on \texttt{unitree\_slam} and \texttt{lidar\_driver} in the App.
\item Do not use App navigation at the same time as API calls.
\end{itemize}
\textbf{Coordinate frame (important)}
\begin{itemize}
\item Origin: MID360-IMU coordinate system.
\item +X forward, +Z up.
\end{itemize}
\textbf{Recommended scope}
\begin{itemize}
\item Static indoor, flat scenes with rich features.
\item X/Y map size under 45 m.
\item Avoid violent movements to prevent localization loss.
\end{itemize}
\end{frame}

\begin{frame}[fragile]{Network check (quick)}
\begin{enumerate}
\item Set your PC to \texttt{192.168.123.XXX} (no IP conflicts).
\item Test connections:
\end{enumerate}
\begin{lstlisting}
ssh unitree@192.168.123.164
ping 192.168.123.120
\end{lstlisting}
\end{frame}

\begin{frame}{Core API IDs}
\small
\scriptsize
\begin{tabular}{lll}
\toprule
Action & API ID & Purpose \\
\midrule
Start mapping & 1801 & Begin SLAM map creation \\
End mapping & 1802 & Save the map (PCD file) \\
Initialize pose & 1804 & Load map + set start pose \\
Pose navigation & 1102 & Navigate to a target pose \\
Pause navigation & 1201 & Pause current nav \\
Resume navigation & 1202 & Resume current nav \\
Close SLAM & 1901 & Stop SLAM service \\
\bottomrule
\end{tabular}
\small
\end{frame}

\begin{frame}[fragile]{Create a map (start → explore → save)}
\textbf{1) Start mapping}
\begin{lstlisting}
{
  "api_id": 1801,
  "data": {
    "slam_type": "indoor"
  }
}
\end{lstlisting}
\textbf{2) Explore slowly}
\begin{itemize}
\item Walk the robot around the space at low speed.
\item Try to close a loop for a clean map.
\end{itemize}
\textbf{3) End mapping and save}
\begin{lstlisting}
{
  "api_id": 1802,
  "data": {
    "address": "/home/unitree/test1.pcd"
  }
}
\end{lstlisting}
Note:
\begin{itemize}
\item Reuse file names like \texttt{test1.pcd} to \texttt{test10.pcd} to avoid filling disk.
\end{itemize}
\end{frame}

\begin{frame}[fragile]{Use a saved map (localize → navigate)}
\textbf{1) Initialize pose with the map}
\begin{lstlisting}
{
  "api_id": 1804,
  "data": {
    "x": 0.0,
    "y": 0.0,
    "z": 0.0,
    "q_x": 0.0,
    "q_y": 0.0,
    "q_z": 0.0,
    "q_w": 1.0,
    "address": "/home/unitree/test1.pcd"
  }
}
\end{lstlisting}
\textbf{2) Send a navigation goal}
\begin{lstlisting}
{
  "api_id": 1102,
  "data": {
    "targetPose": {
      "x": 2.0,
      "y": 0.0,
      "z": 0.0,
      "q_x": 0.0,
      "q_y": 0.0,
      "q_z": 0.0,
      "q_w": 1.0
    },
    "mode": 1
  }
}
\end{lstlisting}
Notes:
\begin{itemize}
\item Target distance must be within 10 m.
\item Robot moves in a straight line.
\item Obstacles should be at least 50 cm tall.
\end{itemize}
\end{frame}

\begin{frame}{Topics to watch (optional, for debugging)}
\small
\scriptsize
\begin{tabular}{ll}
\toprule
Topic & Data \\
\midrule
\texttt{rt/unitree/slam\_mapping/points} & Mapping point cloud \\
\texttt{rt/unitree/slam\_mapping/odom} & Mapping odometry \\
\texttt{rt/unitree/slam\_relocation/points} & Relocation point cloud \\
\texttt{rt/unitree/slam\_relocation/odom} & Relocation odometry \\
\texttt{rt/slam\_info} & Status broadcast (JSON) \\
\texttt{rt/slam\_key\_info} & Task result (JSON) \\
\bottomrule
\end{tabular}
\small
Beginner notes:
\begin{itemize}
\item Start in a simple room before trying corridors or clutter.
\item Keep speed low while building the map.
\item Save the map immediately after a clean loop.
\end{itemize}
\end{frame}

\begin{frame}[plain]
\centering
\vfill
\Huge{Path planning}
\vfill
\end{frame}

\begin{frame}[fragile]{Path planning}
\begin{center}
\includegraphics[width=0.8\linewidth]{./placeholder.jpg}
\end{center}
\textbf{Example goal}
\begin{lstlisting}
{
  "api_id": 1102,
  "data": {
    "targetPose": {
      "x": 2.0,
      "y": 1.0,
      "z": 0.0,
      "q_x": 0.0,
      "q_y": 0.0,
      "q_z": 0.0,
      "q_w": 1.0
    },
    "mode": 1
  }
}
\end{lstlisting}
Beginner notes:
\begin{itemize}
\item Use short, easy goals first (1--2 meters).
\item Confirm localization is stable before sending a goal.
\item If planning fails, remap or reduce map size.
\end{itemize}
\end{frame}

\begin{frame}[plain]
\centering
\vfill
\Huge{Obstacle avoidance}
\vfill
\end{frame}

\begin{frame}{Obstacle avoidance}
\begin{center}
\includegraphics[width=0.85\linewidth,height=0.45\textheight,keepaspectratio]{imgs/ea38cdd7418e4b81852c819a55e7aa2e_1164x1000.jpg}
\end{center}
\begin{itemize}
\item Enable obstacle avoidance in the same navigation task after map initialization.
\item Use the navigation stack to replan when new obstacles appear.
\end{itemize}
Beginner notes:
\begin{itemize}
\item Place one or two obstacles first, then increase complexity.
\end{itemize}
\end{frame}

\begin{frame}[plain]
\centering
\vfill
\Huge{Implementation details (from repo scripts)}
\vfill
\end{frame}

\begin{frame}{Implementation details (from repo scripts)}
\begin{itemize}
\item \texttt{g1/scripts/obstacle\_avoidance/navigate.py} is the orchestration entry point. It initializes DDS, starts \texttt{LocoClient}, spins up \texttt{ObstacleDetector}, creates or loads an \texttt{OccupancyGrid}, then repeatedly plans (A*), smooths, and executes waypoints. It replans when \texttt{front\_blocked()} is true or when a waypoint times out.
\item \texttt{g1/scripts/obstacle\_avoidance/obstacle\_detection.py} subscribes to \texttt{rt/sportmodestate} and reads \texttt{range\_obstacle[4]} plus \texttt{imu\_state.rpy} to compute pose and obstacle directions (front, right, rear, left). It exposes thread-safe getters and a stale-data check.
\item \texttt{g1/scripts/obstacle\_avoidance/create\_map.py} implements a 2D grid with world-to-grid conversion, obstacle marking from range data, and inflation (via binary dilation) for safe planning margins.
\item \texttt{g1/scripts/obstacle\_avoidance/path\_planner.py} runs A* on the inflated grid, then smooths the path using line-of-sight checks and converts it into world-coordinate waypoints spaced by \texttt{--spacing} (default 0.5 m).
\item \texttt{g1/scripts/obstacle\_avoidance/locomotion.py} provides a proportional controller: turn-in-place until heading error is below \texttt{yaw\_tolerance}, then move forward with bounded \texttt{vx} and \texttt{vyaw}. It accepts a \texttt{check\_obstacle} callback so the planner can abort and replan cleanly.
\item \texttt{g1/scripts/obstacle\_avoidance/map\_viewer.py} renders a live OpenCV map (optional \texttt{--viz}) showing inflated cells, planned path, waypoints, and sensor rays for debug sessions.
\end{itemize}
\end{frame}

\end{document}