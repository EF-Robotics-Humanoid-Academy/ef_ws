\documentclass{beamer}
\usepackage[utf8]{inputenc}
\usepackage[T1]{fontenc}
\usepackage{graphicx}
\usepackage{xcolor}
\usepackage{hyperref}
\usepackage{booktabs}
\usepackage{array}
\usepackage{verbatim}

% Define custom colors from the CSS
\definecolor{AccentColor2}{HTML}{FA1C63}
\definecolor{AccentColor1}{HTML}{222222}

% Theme settings - SWAPPED ASSIGNMENTS
\usetheme{default}
\setbeamercolor{palette primary}{bg=AccentColor1,fg=white}
\setbeamercolor{palette secondary}{bg=AccentColor2,fg=white}
\setbeamercolor{palette tertiary}{bg=AccentColor2,fg=white}
\setbeamercolor{title}{bg=AccentColor1,fg=white}
\setbeamercolor{frametitle}{bg=AccentColor1,fg=white}
\setbeamercolor{structure}{fg=AccentColor2}

% Paths for local images
\graphicspath{{../inhalt/}{../inhalt/assets/}}

\title[Day 1: Introduction and Fundamentals]{Day 1: Introduction and Fundamentals}
\author{Unitree G1 Academy}
\date{February 2026}
\logo{\includegraphics[height=1cm]{Logo_1.jpg}}

\begin{document}
\begin{frame}
    \titlepage
\end{frame}

\begin{frame}[plain]
\centering
\vfill
\Huge{Intro}
\vfill
\end{frame}

\begin{frame}{Intro}
\begin{itemize}
\item EF Robotics: Overview of EF Robotics, lab environment, and safety-first operating culture.
\item G1 Masterclass: What you will learn and how hands-on sessions are structured.
\item Course design: Each day builds from simple, safe commands to more advanced behaviors so new learners can program the G1 with confidence.
\item What you will do by the end of Day 1: connect to the robot, send a basic motion command, and understand the safety modes.
\end{itemize}
\end{frame}

\begin{frame}{Technical Specifications (Overview)}
\href{https://docs.quadruped.de/projects/g1/html/\_images/g1\_run.png}{G1 overview run}
\textbf{Body and Dimensions}
\begin{tabular}{ll}
\toprule
Item & Value \\
\midrule
Weight & 35 kg \\
Height & 1270 mm \\
Total Degrees of Freedom & Up to 43 \\
Max Joint Torque & 120 N.m \\
\bottomrule
\end{tabular}
\textbf{Course Schedule (Tag 1 -- Mo 16.02.2026)}
\begin{itemize}
\item 07:30 -- 09:30 Aufbau / Technik
\item 09:30 Doors open Registrierung
\item 10:00 -- 11:30 Schulungsblock
\item 11:30 -- 11:45 Pause
\item 11:45 -- 13:00 Schulungsblock
\item 13:00 -- 14:00 Mittagessen
\item 14:00 -- 15:30 Schulungsblock
\item 15:30 -- 15:45 Pause
\item 15:45 -- 17:00 Schulungsblock
\item 17:00 -- 17:30 Podcast / Testimonials
\item 17:30 -- 17:45 Puffer
\item 17:45 -- 18:15 Tages-Recap
\item 18:15 -- 18:30 Tagesabschluss
\end{itemize}
\textbf{Requirements (Laptop)}
\begin{itemize}
\item RJ45 (Ethernet) or a docking station with RJ45
\item 8-16 GB RAM
\item 250 GB SSD free space
\end{itemize}
\end{frame}

\begin{frame}[plain]
\centering
\vfill
\Huge{Day 1 outline}
\vfill
\end{frame}

\begin{frame}{Day 1 outline}
\begin{itemize}
\item Intro
\item G1 Basics
\item Lunch break
\item Workspace Setup
\item SDK Basics
\end{itemize}
Beginner focus:
\begin{itemize}
\item Today is about safe setup and small, repeatable commands.
\end{itemize}
\end{frame}

\begin{frame}[plain]
\centering
\vfill
\Huge{Lunch Break}
\vfill
\end{frame}

\begin{frame}[plain]
\centering
\vfill
\Huge{G1 Basics}
\vfill
\end{frame}

\begin{frame}{Hardware}
\href{https://www.docs.quadruped.de/projects/g1/html/\_images/g1\_component\_overview.png}{G1 component overview}
\textbf{Actors}
\begin{tabular}{lll}
\toprule
Actor & Role & Notes \\
\midrule
Legs (hip/knee/ankle) & Locomotion & Supports multiple gaits and balance modes \\
Arms/Hands & Manipulation & Depends on arm/hand package installed \\
Waist & Upper-body orientation & Check waist fastener before use \\
\bottomrule
\end{tabular}
Beginner notes:
\begin{itemize}
\item Actors are what move (motors/joints). Sensors are what measure (orientation, depth, distance).
\item Most beginner scripts control only a few joints at a time or use the built-in walking controller.
\item Keep arm motion slow at first so balance and timing are easy to see.
\end{itemize}
\textbf{Sensors}
\begin{tabular}{lll}
\toprule
Sensor & Role & Notes \\
\midrule
IMU & Balance and stability & Used in gait control loops \\
LiDAR & Mapping and navigation & MID-360 supported \\
Depth camera & Perception & RealSense D435i supported \\
\bottomrule
\end{tabular}
\href{https://doc-cdn.unitree.com/static/2024/10/12/b7aab82da80940faa773f213baf13e32\_13364x6401.png}{G1 sensor overview 1}
\href{https://doc-cdn.unitree.com/static/2024/7/30/4afc81d7c48c452aaa4fc078f90a859f\_627x206.png}{G1 sensor overview 2}
\href{https://doc-cdn.unitree.com/static/2024/7/30/31b7d70c4ec1463a8143af70c43f33b9\_592x897.png}{G1 sensor overview 3}
Beginner notes:
\begin{itemize}
\item IMU is the "balance sense"; LiDAR and camera are for mapping and perception.
\item For Day 1, you do not need to process sensor data yet; we only verify they are detected.
\end{itemize}
\end{frame}

\begin{frame}{Software}
\href{https://doc-cdn.unitree.com/static/2024/9/18/5da5c8fdc8f84b59aa3f2f5d45add0e4\_8000x6106.jpg}{G1 system architecture}
Beginner notes:
\begin{itemize}
\item You will use high-level RPC clients first (\texttt{LocoClient}, \texttt{ArmActionClient}) before any low-level control.
\item The robot can keep balancing while you send arm commands, but we start in safe, static poses.
\end{itemize}
\end{frame}

\begin{frame}{Safety checklist}
\href{https://doc-cdn.unitree.com/static/2024/9/29/236fa93a8fae4eaa8815004f42e87ede\_1065x1419.jpg}{Debug mode on controller}
Beginner notes:
\begin{itemize}
\item Always start in a clear area with hard, flat ground and no obstacles within 1--2 meters.
\item For low-level commands, confirm \texttt{Debug mode} is enabled before sending any motion.
\item The robot does not stop for obstacles when you send low-level velocity commands.
\item Know how long a command will run (duration is part of the command, not automatic).
\end{itemize}
\textbf{Modes and meaning}
\begin{tabular}{ll}
\toprule
State/Mode & Description \\
\midrule
Zero-Torque & Motors unpowered for safe handling \\
Damping & Soft resistance for safe positioning \\
Balanced Stand & Stable idle stance for transitions \\
\bottomrule
\end{tabular}
\textbf{Joint limits and constraints}
\begin{tabular}{ll}
\toprule
Item & Description \\
\midrule
Joint limits & Keep motions inside safe mechanical ranges \\
Starting posture & Use flat hard ground for transitions \\
Debug mode & Required before sending low-level commands \\
\bottomrule
\end{tabular}
\end{frame}

\begin{frame}{Remote Controller}
\href{https://marketing.unitree.com/article/en/G1/Remote\_Control.html}{Remote controller and debug mode}
\begin{tabular}{ll}
\toprule
Keybinding & Description \\
\midrule
L2 + R2 & Enter debug mode for low-level control \\
Start/Stop & Safe motion start/stop sequence \\
Mode toggle & Switch between motion modes \\
\bottomrule
\end{tabular}
Beginner notes:
\begin{itemize}
\item Practice entering and exiting debug mode before running scripts.
\item Keep one person focused on the controller while another runs code.
\end{itemize}
Controller documentation: \href{https://marketing.unitree.com/article/en/G1/Remote\_Control.html}{https://marketing.unitree.com/article/en/G1/Remote\_Control.html}
\end{frame}

\begin{frame}{Unitree App}
\href{https://marketing.unitree.com/article/en/G1/Remote\_Control.html}{Unitree app overview}
Beginner notes:
\begin{itemize}
\item The app is used for quick checks (camera preview, status, battery).
\item We use the app for verification, not for programming.
\end{itemize}
App documentation: \href{https://www.unitree.com/app/g1}{https://www.unitree.com/app/g1}
\end{frame}

\begin{frame}[plain]
\centering
\vfill
\Huge{Workspace Setup}
\vfill
\end{frame}

\begin{frame}{Workspace Setup}
\begin{itemize}
\item Ubuntu 24.04: Standard OS for labs and SDK tools.
\item Unitree repos: \texttt{unitree\_sdk2\_python}, \texttt{unitree\_sdk2}, \texttt{unitree\_mujoco}, \texttt{unitree\_ros}, \href{https://github.com/ahmedgalaief/ef\_ws.git}{https://github.com/ahmedgalaief/ef\_ws.git}.
\item CycloneDDS setup: Configure DDS for reliable pub/sub on the LAN.
\item Python environment: Create a dedicated venv or conda env for SDKs.
\item AI CLIs: Use \texttt{codex}, \texttt{claude code}, \texttt{gemini cli} for scripting support.
\end{itemize}
Beginner steps:
\begin{itemize}
\item Verify the robot and laptop are on the same subnet.
\item Identify your network interface name (e.g., \texttt{enp3s0}) before running SDK examples.
\item Start with small test scripts first, then expand.
\end{itemize}
\end{frame}

\begin{frame}{AI CLI Agents (Codex and Friends)}
\begin{center}
\includegraphics[width=0.8\linewidth]{./codex.jpg}
\end{center}
Use AI CLI agents to speed up repetitive work: generate small scripts, explain SDK snippets, and draft troubleshooting steps.
Beginner tips:
\begin{itemize}
\item Ask for tiny, specific tasks (e.g., "write a 3-second walk command").
\item Verify the output before running it on the robot.
\item Keep a human in the loop for safety checks and parameter review.
\end{itemize}
\end{frame}

\begin{frame}[plain]
\centering
\vfill
\Huge{SDK Basics}
\vfill
\end{frame}

\begin{frame}{SDK Basics}
\begin{itemize}
\item Repo structure: \texttt{unitree\_sdk2} for C++, \texttt{unitree\_sdk2\_python} for Python.
\item Ethernet connection: Robot IP and interface name are required for SDK examples.
\item Basic example: \texttt{g1\_loco\_client\_example.py} for first motion test.
\end{itemize}
Beginner steps:
\begin{itemize}
\item Put the robot in balanced stand before any motion command.
\item Run a short, low-speed test (e.g., 0.1--0.2 m/s for 2--3 seconds).
\item Stop the robot explicitly after each motion command.
\end{itemize}
\end{frame}

\begin{frame}[fragile]{Motion Switcher Service Interface}
MotionSwitcherClient lets you release and switch motion control modes via RPC and enter your own debug mode.
\textbf{Include (C++)}
\begin{verbatim}
#include <unitree/robot/b2/motion_switcher/motion_switcher_client.hpp>
\end{verbatim}
\textbf{Core functions}
\begin{tabular}{lll}
\toprule
Function & Prototype & Summary \\
\midrule
\texttt{CheckMode} & \texttt{int32\_t CheckMode(std::string\& form, std::string\& name)} & Detects current form and motion control mode. \\
\texttt{SelectMode} & \texttt{int32\_t SelectMode(const std::string\& name)} & Selects a motion control mode. \\
\texttt{ReleaseMode} & \texttt{int32\_t ReleaseMode()} & Releases current motion mode. \\
\bottomrule
\end{tabular}
Notes:
\begin{itemize}
\item \texttt{form} is \texttt{"0"} for Standard Form, \texttt{"1"} for Wheel-Foot Form.
\item \texttt{name} is the motion control mode name as defined by the system.
\end{itemize}
\textbf{Interface error codes}
\begin{tabular}{lll}
\toprule
Error number & Description & Remarks \\
\midrule
7001 & Request parameter error & Server return \\
7002 & Service busy, retry again please & Server return \\
7004 & Unsupport mode name & Server return \\
7005 & Internal command execute error & Server return \\
7006 & Check command execute error & Server return \\
7007 & Switch command execute error & Server return \\
7008 & Release command execute error & Server return \\
7009 & Customize config set error & Server return \\
\bottomrule
\end{tabular}
\textbf{Usage example (C++)}
\begin{verbatim}
MotionSwitcherClient msc;
msc.Init();

std::string form;
std::string mode;
msc.CheckMode(form, mode);

msc.SelectMode("stand");
// ...
msc.ReleaseMode();
\end{verbatim}
\end{frame}

\begin{frame}[plain]
\centering
\vfill
\Huge{Implementation details (from repo scripts)}
\vfill
\end{frame}

\begin{frame}{Implementation details (from repo scripts)}
\begin{itemize}
\item \texttt{g1/scripts/safety/hanger\_boot\_sequence.py} implements a repeatable boot path from hanger to balanced stand. It calls \texttt{ChannelFactoryInitialize(0, iface)}, creates a \texttt{LocoClient}, then sequences \texttt{Damp} → \texttt{SetFsmId(4)} → \texttt{SetStandHeight} sweep until \texttt{FSM mode == 0} (feet loaded) → \texttt{BalanceStand} → \texttt{Start}. If the robot is already in FSM 200 with loaded feet, it returns early to avoid re-running the sequence.
\item The same script polls \texttt{ROBOT\_API\_ID\_LOCO\_GET\_FSM\_ID} and \texttt{ROBOT\_API\_ID\_LOCO\_GET\_FSM\_MODE} to confirm the current state and to decide whether another height sweep is needed.
\item \texttt{g1/scripts/safety/keyboard\_controller.py} provides a minimal teleop loop with \texttt{pynput} for key-hold detection and sends \texttt{Move(vx, vy, omega, continous\_move=True)} at 10 Hz. Key mapping is W/S for forward/back, A/D for yaw, Q/E for lateral, Space for stop, \texttt{Z} for \texttt{Damp}, and \texttt{Esc} for \texttt{StopMove} + \texttt{ZeroTorque}.
\item \texttt{g1/scripts/unsorted/g1\_loco\_client\_example.py} is the Day 1 interactive test harness. It initializes DDS, then lets you trigger actions like \texttt{Damp}, \texttt{Move}, \texttt{WaveHand}, \texttt{ZeroTorque}, and posture changes via a simple CLI menu.
\end{itemize}
\end{frame}

\end{document}